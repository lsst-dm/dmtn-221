\documentclass[DM,authoryear,toc]{lsstdoc}
% lsstdoc documentation: https://lsst-texmf.lsst.io/lsstdoc.html
\input{meta}

% Package imports go here.

% Local commands go here.

%If you want glossaries
%\input{aglossary.tex}
%\makeglossaries

\title{Periodicity Analysis in Alert Production}

% Optional subtitle
% \setDocSubtitle{A subtitle}

\author{%
Anastasios Tzanidakis, Eric Bellm
}

\setDocRef{DMTN-221}
\setDocUpstreamLocation{\url{https://github.com/lsst-dm/dmtn-221}}

\date{\vcsDate}

% Optional: name of the document's curator
% \setDocCurator{The Curator of this Document}

\setDocAbstract{%
The baselined timeseries features to be computed in Alert Production include a Lomb-Scargle periodogram.  We assess the computational and scientific performance of several configurations on simulated alert data.
}


% Change history defined here.
% Order: oldest first.
% Fields: VERSION, DATE, DESCRIPTION, OWNER NAME.
% See LPM-51 for version number policy.
\setDocChangeRecord{%
  \addtohist{1}{YYYY-MM-DD}{Unreleased.}{Anastasios Tzanidakis, Eric Bellm}
}


\begin{document}

% Create the title page.
\maketitle
% Frequently for a technote we do not want a title page  uncomment this to remove the title page and changelog.
% use \mkshorttitle to remove the extra pages

% ADD CONTENT HERE
% You can also use the \input command to include several content files.

\section{Note}
\textcolor{red}{Please note that this DM-technote is currently work in progress, and publicly available. The final findings will be pushed to the DMTN-221 repository in late November of 2022.}


\section{Motivation}
The characterization of periodicity from time-series is a fundamental constraint to numerous astrophysical applications. Phenomenologically, estimating the periodicity and its significance can shed light on stellar pulsation theory \citep{Antonello:Antonello81}, distance estimation and mapping of the Galaxy through the period-luminosity relationship \citep{Skowron:Skowron2019}, constraint fundamental parameters of stellar binaries \citep{Farinella:Farinella1979} and stellar rotation \citep{Walkowicz:Walkowicz13}. In the recent decade, the use of periodicity has also been extensively used as a feature to classify variable phenomena \citep{Richards:R13} across the HR diagram. 


Previous work from \cite{2012AJ....144....9O} demonstrate the concept of injection-recovery test for synthetically generated RR Lyrae templates from SDSS using the op$\_$sim version 1.29 cadence strategy. However, no study to date has demonstrated the recovery period distribution of more complex periodic phenomena (i.e eclipsing binaries, quasi-periodic AGN). Here we perform an end-to-end, injection and recovery analysis for multi-band time series to asses the statistical significance of period finding in the context of AP. 


\section{Synthetic Light Curves}
In this study we use the training set Extended LSST Astronomical Time-series Classification Challenge (ELAsTiCC) light curves with a 12 month history that will best mimic the calculations performed for AP. In short, ELAsTiCC is an end-to-end real-time pipeline for generating mock photometric alerts at the expected Legacy Survey of Time and Space (LSST) rate. The photometric alerts will be distributed in real-time to brokers to benchmark classification algorithms. The ELAsTiCC uses the v.1.7 cadence strategy from op$\_$sims. A single detection is based on the DIA performance from DC2. Each light curve include photometric noise using Poisson noise that includes the equivalent area, background noise per unit area, sky and CCD read noise. Most importantly, the ELAsTiCC light curves simulate force photometry alerts which we recommend all time series features to be calculated upon. 
	
	Here we present two classes of periodic sources: RR Lyrae (RRL) and eclipsing binaries (EB). Both RRL's and EB's represent both the extrema of simple and complex periodic sources that will throughly test the underlying ability of the periodogram to find the correct period. Currently, the eclipsing binary sample from ELAsTiCC that was generated from open-source Python package (ELISa). the only contains a few hundred simulated binaries. To maximize the use different eclipsing binary stars sampled at different phases, each light curve over a three year photometric history is sampled at some random time and ensures that the maximum baseline is less than 12 months. In Figure \ref{fig:light_curve_demo} we show the typical multi-band AP light curve sampled from ELAsTiCC with uncertainties. A detailed Jupyter notebook interacting and fetching the ELAsTiCC light curves can be found here. 

\begin{figure*}
  \includegraphics[width=0.9\textwidth]{figures/lightcurve_demo.pdf}
  \centering 
  \caption{The above panels demonstrate the multi-band light curves simulated from ElasTiCC using RR lyrae (left panel) and eclipsing binaries (right panel). The first column of each panel shows the observed light curves, and the second the phase folded light curves at the correct period after 12 months.}
   \label{fig:light_curve_demo}
\end{figure*}


\section{Injection-Recovery Testing}
We consider two models of periodic phenomena: RR lyrae and eclipsing binaries ranging from a periods of 0.2 days to 10 days. All computations of the floating-mean Lomb-Scargle periodogram are using the open-source Python package `gatspy` \citet{VanderPlas:VP2015}. 


\subsection{Single Band Lomb-Scargle Periodogram}

We first consider the case of single-band detections from AP. Here we aim to answer if the 12 month AP history in a single-band light curve is enough to recover periods. We consider the canonical floating mean to estimate the periodogram of 10,000 alert light curves in the $r$-band for three Fourier component modes. In Figure \ref{fig:single_band_lsp} we show the results of this analysis for RRL's (rop row) and EB's (bottom row). For each light curve model, we randomly sample a light curve and compute the Lomb-Scargle periodogram using a heuristic period grid with an oversampling factor of 2 and Nyquist frequency of 30 ($\sim$0.03 days) to account for the fast periodic phenomena. We extrapolated the highest power period from the periodogram and compare it to the true period. In Figure \ref{fig:single_band_lsp} we also include the number of $r$-band detections for each light curve. As expected, we did not find any preference of an optical single-band photometric filter that outperformed the others. While the RR Lyrae exhibit at larger Fourier modeling modes more aliasing, overall the recovery fraction of correctly identified periods is larger compared to the EBs. For both light curve models, we consistently found in each model configuration that the average correctly identified period coincides with a larger number of detections. Overall we find consistent results for the RR Lyrae done by a similar analysis  by \citet{VanderPlas:VP2015}. The true challenge of this approach comes from the more complex light curves such as the EBs. 


% Todo
\begin{itemize}
\item Injected vs recovered period plots eclipsing binaries and RRL
\item Fraction of correctly recovered periods using single-band Lomb-Scargle (for N=1...3 Fourier components)  
\end{itemize}

\begin{figure*}
  \includegraphics[width=0.9\textwidth]{figures/singleband_lsp.pdf}
  \centering 
  \caption{We show the injected (true period) versus recovered highest peak in the periodogram for a sample of 1000 $r$-band detections from RR lyrae and eclipsing binaries. Each column shows the period recovery for N=1 to N=3 Fourier components when computing the Lomb-Scargle periodogram.}
  \label{fig:single_band_lsp}
\end{figure*}

% Conclusions regarding the single-band Lomb Scargle
Even with a low completeness, computing the Lomb-Scargle periodogram on single-band time series alerts, this will likely outperform any other known survey due to the large survey volume. Single-band Lomb-Scargle implementations are also faster by a factor of $\sim$5 due to the lower average number of detections per filter. Given that only roughly 17 percent of each bandpass filter will be available, the single-band Lomb-Scargle misses by a higher margin the underlying true period. We now turn our attention to the multi-band Lomb-Scargle.

\subsection{Multi-Band Lomb-Scargle Periodogram}

Next, we test the multi-band Lomb-Scargle periodogram implementation adapted on \textit{gatspy} to our ELAsTiCC light curves.

% TODO
\begin{itemize}
\item Motivation and mathematical explanation of multi-band Lomb Scargle 
\item Injected vs recovered period plots eclipsing binaries and RRL
\item Fraction of correctly recovered periods using single-band Lomb-Scargle (for N=1...3 Fourier components)  
\end{itemize}


\begin{figure*}
  \includegraphics[width=0.9\textwidth]{figures/multi_lsp_rectest.pdf}
  \centering 
  \caption{Multi-band injection-recovery test for 5000 RR Lyrae (top row) and eclipsing binaries (bottom row). Each column title includes the number of Fourier base and band terms used to compute the Lomb-Scargle periodogram. Additionally we color code each recovered period by the number of detections. The overlaid curved lines represent the n=1 aliasing, while the straight lines represent the n=1,2 harmonics.}
\end{figure*}


\subsection{Peak Significance Metric}

% TODO
\begin{itemize}
\item Bootstraping results 
\item Bootstrap results correlation with Baluev approximation
\item Application to other survey data (i.e SDSS)
\end{itemize}


\section{Timing Analysis}
In this section we provide information regarding the scalability and run-time of both single-band and multi-band Lomb Scargle periodogram. 

\begin{figure*}
  \includegraphics[width=0.9\textwidth]{figures/run_time_analysis.pdf}
  \centering 
  \caption{Average run time per number of $ugrizy$ light curve detections. Each column represents the model complexity of each multi-band Lomb Scargle periodogram. Within each panel, we use a heuristically determined period grid by varying the oversampling factor and the the Nyquist frequency.  Finally, for each successive run we color code the fractional error compared from the true period.}
\end{figure*}

% TODO
\begin{itemize}
\item Run time as a function of Fourier components (single band)
\item Run time as a function of Fourier components (multi-band)
\item Run time as a function of Nyquist frequency and oversampling factor
\item Discussion on how many periodic sources we expect to find
\item What recommendations do we make from number of detections and numbers of detections per filter
\item Test new Astropy multi-band Lomb Scargle implementation, on timing analysis and results 
\end{itemize}


\appendix
% Include all the relevant bib files.
% https://lsst-texmf.lsst.io/lsstdoc.html#bibliographies
\section{References} \label{sec:bib}
\bibliography{local,lsst,lsst-dm,refs_ads,refs,books}

% Make sure lsst-texmf/bin/generateAcronyms.py is in your path
\section{Acronyms} \label{sec:acronyms}
\addtocounter{table}{-1}
\begin{longtable}{p{0.145\textwidth}p{0.8\textwidth}}\hline
\textbf{Acronym} & \textbf{Description}  \\\hline

AGN & active galactic nuclei \\\hline
AP & Alert Production \\\hline
API & Application Programming Interface \\\hline
CCD & Charge-Coupled Device \\\hline
DC2 & Data Challenge 2 (DESC) \\\hline
DIA & Difference Image Analysis \\\hline
DM & Data Management \\\hline
DMTN & DM Technical Note \\\hline
LSST & Legacy Survey of Space and Time (formerly Large Synoptic Survey Telescope) \\\hline
SDSS & Sloan Digital Sky Survey \\\hline
SNR & Signal to Noise Ratio \\\hline
\end{longtable}






\end{document}
