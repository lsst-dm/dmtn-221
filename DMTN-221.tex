\documentclass[DM,authoryear,toc]{lsstdoc}
% lsstdoc documentation: https://lsst-texmf.lsst.io/lsstdoc.html
\input{meta}

% Package imports go here.

% Local commands go here.

%If you want glossaries
%\input{aglossary.tex}
%\makeglossaries

\title{Periodicity Analysis in Alert Production}

% Optional subtitle
% \setDocSubtitle{A subtitle}

\author{%
Anastasios Tzanidakis
}

\setDocRef{DMTN-221}
\setDocUpstreamLocation{\url{https://github.com/lsst-dm/dmtn-221}}

\date{\vcsDate}

% Optional: name of the document's curator
% \setDocCurator{The Curator of this Document}

\setDocAbstract{%
The baselined timeseries features to be computed in Alert Production include a Lomb-Scargle periodogram.  We assess the computational and scientific performance of several configurations on simulated alert data.
}


% Change history defined here.
% Order: oldest first.
% Fields: VERSION, DATE, DESCRIPTION, OWNER NAME.
% See LPM-51 for version number policy.
\setDocChangeRecord{%
  \addtohist{1}{YYYY-MM-DD}{Unreleased.}{Anastasios Tzanidakis, Eric Bellm}
}


\begin{document}

% Create the title page.
\maketitle
% Frequently for a technote we do not want a title page  uncomment this to remove the title page and changelog.
% use \mkshorttitle to remove the extra pages

% ADD CONTENT HERE
% You can also use the \input command to include several content files.

\section{Motivation}
The characterization of periodicity from time-series is a fundamental constraint to numerous astrophysical applications. Phenomenologically, estimating the periodicity and its significance can shed light on stellar pulsation theory \citep{Antonello:Antonello81}, distance estimation and mapping of the Galaxy through Cepeheids, Miras and RR Lyrae \citep{Skowron:Skowron2019}, constraint fundamental parameters of stellar binaries \citep{Farinella:Farinella1979}, and stellar rotation \citep{Walkowicz:Walkowicz13}. In the recent decade, the use of periodicity has also been extensively used as a feature to classify transient phenomena \citep{Richards:R13}.



\section{Synthetic Light Curves}

\begin{figure*}
  \includegraphics[width=0.9\textwidth]{figures/lightcurve_demo.pdf}
  \centering 
  \caption{The above panels demonstrate the multi-band light curves simulated from ElasTiCC using RR lyrae (left panel) and eclipsing binaries (right panel). The first column of each panel shows the observed light curves, and the second the phase folded light curves at the correct period after 12 months.}
  \label{fig:comp}
\end{figure*}



\section{Injection-Recovery Testing}

We consider two models of periodic phenomena: RR lyrae and eclipsing binaries ranging from a periods of 0.2 days to 10 days. All computations of the floating-mean Lomb-Scargle periodogram are using the out-of-box open-source Python package `gatspy` \citet{VanderPlas:VP2015}. 



\subsection{Single Band Lomb-Scargle Periodogram}

% TODO
\begin{itemize}
\item Injected vs recovered period plots EB and RRL
\item Fraction of correctly recovered periods using single-band Lomb-Scargle (for N=1...3 Fourier components)  
\end{itemize}

\begin{figure*}
  \includegraphics[width=0.9\textwidth]{figures/singleband_lsp.pdf}
  \centering 
  \caption{We show the injected (true period) versus recovered highest peak in the periodogram for a sample of 1000 RR lyrae and eclipsing binaries. Each column shows the period recovery for N=1 to N=3 Fourier components when computing the Lomb-Scargle periodogram.}
  \label{fig:comp}
\end{figure*}




\subsection{Multi-Band Lomb-Scargle Periodogram}


% TODO
\begin{itemize}
\item Motivation and mathematical explanation of multi-band Lomb Scargle 
\item Injected vs recovered period plots EB and RRL
\item Fraction of correctly recovered periods using single-band Lomb-Scargle (for N=1...3 Fourier components)  
\end{itemize}


\subsection{Peak Significance Metric}

% TODO
\begin{itemize}
\item Bootstraping results 
\item Bootstrap results correlation with Baluev approximation
\item Application to other survey data (i.e SDSS)
\end{itemize}


\section{Timing Analysis}

% TODO
\begin{itemize}
\item Run time as a function of Fourier components (single band)
\item Run time as a function of Fourier components (multi-band)
\item Run time as a function of Nyquist frequency and oversampling factor
\end{itemize}





\appendix
% Include all the relevant bib files.
% https://lsst-texmf.lsst.io/lsstdoc.html#bibliographies
\section{References} \label{sec:bib}
\renewcommand{\refname}{} % Suppress default Bibliography section
\bibliography{local,lsst,lsst-dm,refs_ads,refs,books}

% Make sure lsst-texmf/bin/generateAcronyms.py is in your path
\section{Acronyms} \label{sec:acronyms}
\addtocounter{table}{-1}
\begin{longtable}{p{0.145\textwidth}p{0.8\textwidth}}\hline
\textbf{Acronym} & \textbf{Description}  \\\hline

AGN & active galactic nuclei \\\hline
AP & Alert Production \\\hline
API & Application Programming Interface \\\hline
CCD & Charge-Coupled Device \\\hline
DC2 & Data Challenge 2 (DESC) \\\hline
DIA & Difference Image Analysis \\\hline
DM & Data Management \\\hline
DMTN & DM Technical Note \\\hline
LSST & Legacy Survey of Space and Time (formerly Large Synoptic Survey Telescope) \\\hline
SDSS & Sloan Digital Sky Survey \\\hline
SNR & Signal to Noise Ratio \\\hline
\end{longtable}

% If you want glossary uncomment below -- comment out the two lines above
%\printglossaries





\end{document}
